%%%%%%%%%%%%%%%%%%%%%%%%%%%%%%%%%%%%

\section{Case study}

%%%%%%%%%%%%%%%%%%%%%%%%%%%%%%%%%%%%

\begin{frame}
\frametitle{Treating Chronic Fatigue Syndrome}

\begin{itemize}

\item Objective: Evaluate the effectiveness of cognitive-behavior therapy for chronic fatigue syndrome.

\item Participant pool: 142 patients who were recruited from referrals by primary care physicians and consultants to a hospital clinic specializing in chronic fatigue syndrome.

\item Actual participants: Only 60 of the 142 referred patients entered the study. Some were excluded because they didn't meet the diagnostic criteria, some had other health issues, and some refused to be a part of the study.

\end{itemize}

\ct{Deale et. al. \textit{Cognitive behavior therapy for chronic fatigue syndrome: A randomized controlled trial}. The American Journal of Psychiatry 154.3 (1997).}

\end{frame}

%%%%%%%%%%%%%%%%%%%%%%%%%%%%%%%%%%%%

\begin{frame}
\frametitle{Study design}

\begin{itemize}

\item Patients randomly assigned to treatment and control groups, 30 patients in each group:
\begin{itemize}
\item \hl{Treatment}: Cognitive behavior therapy -- collaborative, educative, and with a behavioral emphasis. Patients were shown on how activity could be increased steadily and safely without exacerbating symptoms.
\item \hl{Control:} Relaxation -- No advice was given about how activity could be increased. Instead progressive muscle relaxation, visualization, and rapid relaxation skills were taught.
\end{itemize}

\end{itemize}

\end{frame}

%%%%%%%%%%%%%%%%%%%%%%%%%%%%%%%%%%%%

\begin{frame}
\frametitle{Results}

The table below shows the distribution of patients with good outcomes at 6-month follow-up. Note that 7 patients dropped out of the study: 3 from the treatment and 4 from the control group.

\begin{center}
\begin{tabular}{ll  cc c} 
			&				& \multicolumn{2}{c}{\textit{Good outcome}} \\
\cline{3-4}
			&							& Yes 	& No 	& Total	\\
\cline{2-5}
							&Treatment 	& 19	 	& 8		& 27 	\\
\raisebox{1.5ex}[0pt]{\textit{Group}}	&Control		& 5	 	& 21	 	& 26 \\
\cline{2-5}
							&Total		& 24		& 29		& 53
\end{tabular}
\end{center}

\pause

\begin{itemize}

\item Proportion with good outcomes in treatment group:
\[ 19 / 27 \approx 0.70 \rightarrow 70\% \]

\pause

\item Proportion with good outcomes in control group:
\[ 5 / 26 \approx 0.19 \rightarrow 19\% \]

\end{itemize}

\end{frame}

%%%%%%%%%%%%%%%%%%%%%%%%%%%%%%%%%%%%

\begin{frame}
\frametitle{Understanding the results}

\dq{Do the data show a ``real" difference between the groups?}

\pause

\begin{itemize}

\item Suppose you flip a coin 100 times. While the chance a coin lands heads in any given coin flip is 50\%, we probably won't observe exactly 50 heads. This type of fluctuation is part of almost any type of data generating process.

\item The observed difference between the two groups (70 - 19 = 51\%) may be real, or may be due to natural variation.

\item Since the difference is quite large, it is more believable that the difference is real.

\item We need statistical tools to determine if the difference is so large that we should reject the notion that it was due to chance.

\end{itemize}

\end{frame}

%%%%%%%%%%%%%%%%%%%%%%%%%%%%%%%%%%%%

\begin{frame}
\frametitle{Generalizing the results}

\dq{Are the results of this study generalizable to all patients with chronic fatigue syndrome?}

\pause

These patients had specific characteristics and volunteered to be a part of this study, therefore they may not be representative of all patients with chronic fatigue syndrome. While we cannot immediately generalize the results to all patients, this first study is encouraging. The method works for patients with some narrow set of characteristics, and that gives hope that it will work, at least to some degree, with other patients.


\end{frame}

%%%%%%%%%%%%%%%%%%%%%%%%%%%%%%%%%%%%
